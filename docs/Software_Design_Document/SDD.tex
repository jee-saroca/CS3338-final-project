\documentclass[12pt]{article}
\usepackage{geometry}
\usepackage{fancyhdr}
\usepackage{graphicx}
\usepackage{titling}

\geometry{a4paper, margin=1in}
\graphicspath{ {../../assets/SDD_images/} }

\title{EagleNAV\\
Software Design Document}
\author{
	Taiten McKiver\\
	\and
	Brandon Polanco\\
	\and
	Jesse Saroca\\
	\and
	Thomas Stender
	}
\date{November 2025}

\setlength{\droptitle}{6cm}

\pagestyle{fancy}
\fancyhead[L]{EagleNAV SDD}
\fancyhead[R]{Page \thepage}
\fancyfoot[C]{}

\begin{document}

\begin{titlepage}
\maketitle
\thispagestyle{empty}
\end{titlepage}

\thispagestyle{empty}
\tableofcontents
\newpage

\section*{Version History}
\addcontentsline{toc}{section}{Version History}
\begin{table}[ht]
    \centering
    \begin{tabular}{|c|c|c|c|}
    \hline
    \textbf{User} & \textbf{Date} & \textbf{Reason for Changes} & \textbf{Version}\\
    \hline
         Jesse Saroca &  12/5/25 &  Update for snapshot 1 & 1.0 \\
     \hline
         Jesse Saroca &  12/11/25 &  Update for snapshot 2 & 1.1 \\
         
    \hline
    \end{tabular}
\end{table}
\newpage

% SECTION 1
\section{Introduction}

\subsection{Purpose}
The purpose of this document is to provide a detailed description of the EagleNAV. This document will outline the system architecture, user interface, and intended audience for the application.

\subsection{Intended Audience}
The intended audience for this document includes:
\begin{itemize}
	\item Software Developers: This document will allow people to understand the system architecture and application UI of the software to understand and build on what was already created.
	\item Project Managers: This document will allow people to understand how the architecture's components interact with each other in order to better plan tasks and resources.
	\item Testers: This document will allow people to understand the architecture before making test cases that will search for all possible bugs and issues before being fully released. 
	\item Stakeholders: This document will people to understand what is being built in the system architecture to have more faith in the project.
\end{itemize}

\subsection{Overview}
The EagleNAV will be a mobile application that allows users to use their phone camera to navigate through their college campuses. The application will provide a user-friendly interface and clear instructions that will guide the user. 

% SECTION 2
\section{System Architecture}
\includegraphics{dfd_1}
\subsection{Workflow}
The user workflow of the EagleNAV will consist of:
\begin{itemize}
	\item User authorization to use location and camera
	\item User submission of destination choice
	\item EagleNAV sends a routing request
	\item EagleNAV directs user towards their destination
\end{itemize}

\subsection{App Breakdown}
The Accessibility Campus Navigation System is divided into several key areas:
\begin{itemize}
	\item Route Planning Subsystem: Uses the local routing server and Route Planner Engine provide the assembled route. 
	\item Guidance Subsystem: Uses sensors to give a consistent direction to the inputted location.
	\item Alerts Subsystem: Uses campus records to indicate obstructions in routes to find a new faster route.
	\item Feedback Subsystem: User inputted feedback reports.
\end{itemize}
\includegraphics{system_architecture}
\subsection{Data Flow - New Audit}

% SECTION 3
\section{User Interface}

\subsection{How to Use}
\begin{itemize}
	\item Map: Includes a search bar and current location.
	\item Camera: Find the navigation arrow in the augmented reality space.
	\item Favorites: Saved places.
	\item Alerts: List of active campus impacts in order to re-route user.
	\item Settings: Change text size, contrast, haptic intensity, or captions.
\end{itemize}
\subsection{Database Explanation}
The database for EagleNAV is stored in json files:
\begin{itemize}
	\item bookmarks.json - stores saved routes
	\item points.jeojson - stores POI and custom landmarks
\end{itemize}

\section*{Glossary}
\addcontentsline{toc}{section}{Glossary}
\begin{description}
    \item[\textbf{AR}] - Augmented Reality
    \item[\textbf{NAV}] - Navigation 
    \item[\textbf{POI}] - Point of Interest
\end{description}

\addcontentsline{toc}{section}{References}
\begin{thebibliography}{9}
     Ascent - Project. (2025-2026). Cysun.org. https://ascent.cysun.org/project/project/view/248
\end{thebibliography}

\end{document}