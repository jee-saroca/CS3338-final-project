\documentclass{article}

\usepackage{geometry}
\usepackage{indentfirst}
\usepackage{fancyhdr}

\pagestyle{fancy}
\geometry{a4paper, margin=1in}

\begin{document}

\title{Read Me}
\author{Brandon Polano}
\date{December 4th, 2025}

\fancyhf{}
\fancyhead[C]{README}
\fancyfoot[C]{\thepage}

\section{Jira Link: }
https://cs3338-group-15.atlassian.net/jira/software/projects/CNUAR/boards/100

\section{Formal objective breakdown: }
Eagle Nav's goal is to give guests, teachers, staff, and students who depend on clear, barrier-free paths around campus an accessible navigation tool. 
To guarantee safe and effective travel, the system places a high value on usability, inclusive design, and the addition of current campus updates. Through the use of a multimodal guiding system that includes visual instructions, sound cues, and haptic feedback, the program accommodates a variety of sensory needs.
Eagle Nav also seeks to empower users with features like customizable preferences, stored routes, and the option to report accessibility problems on campus.

\section{Goals/reason why your software matters}
Eagle Nav is important because it immediately improves people with disabilities or restricted mobility, independence, and campus inclusivity. Without specific assistance, navigating many campuses can be difficult due to major obstacles like stairs, steep paths, or inaccessible buildings. Eagle Nav bridges the gap between conventional navigation apps and users practical accessibility demands by giving priority to accessible routes and providing real time alerts about obstructions. Through dependable and intelligent route planning, the software enables equal navigation for all, raises awareness of accessibility issues, improves campus safety, and improves the user experience overall.


\section{How to download or access, etc.}
Eagle Nav is designed as a cross-platform mobile application built with Flutter and is intended 
to run on both iOS and Android devices. Accessing and installing the application can be done 
through the following methods:

\begin{itemize}
    \item \textbf{App Stores:} The most accessible method is through public distribution on 
    the Apple App Store and Google Play Store, where the app can be located by searching 
    ``Eagle Nav.''
    
    \item \textbf{Institutional Distribution:} For early testing phases or controlled rollouts,
    the app may be distributed through the university’s internal system, such as a Mobile Device
    Management (MDM) platform or secure internal download link.
    
    \item \textbf{First-Time Setup:} After installing the application, the user will be prompted 
    to grant permissions such as Location Access and (optionally) Camera Access for AR navigation.
    Once permissions are granted, the app provides a short on boarding tutorial and allows users 
    to sign in with their campus Single Sign-On (SSO) credentials if needed.
    
    \item \textbf{Offline and Accessibility Features:} Eagle Nav offers offline caching for 
    important campus map segments, accessibility settings (such as voice guidance or high-contrast 
    mode), and optional add-ons depending on user preference.
\end{itemize}

\end{document}