\documentclass[12pt]{article}
\usepackage{geometry}
\usepackage{fancyhdr}
\usepackage{graphicx}
\usepackage{titling}

\geometry{a4paper, margin=1in}

\title{Snapshot Objectives}

\author{
	Taiten McKiver\\
	\and
	Brandon Polanco\\
	\and
	Jesse Saroca\\
	\and
	Thomas Stender
	}
	
\date{November 2025}
	
\setlength{\droptitle}{6cm}

\pagestyle{fancy}

\fancyhead[L]{Snapshot Objectives}
\fancyhead[R]{Page \thepage}
\fancyfoot[C]{}

\begin{document}

\begin{titlepage}
\maketitle
\thispagestyle{empty}
\end{titlepage}

\section*{Start Objective}
The starting objective of the Campus NAV will be to set up the user interface and basic backend pieces for an Android application. We will need to begin with the following:
\begin{itemize}
\item Developing a user-friendly interface on the Android mobile application.
\item Implementing an Augmented Reality software.
\item Implementing a database to store locations and routes around campus.
\end{itemize}

\section*{Checkpoint 1}
The primary objective for Checkpoint 1 is to start getting footage of the campus's major locations and routes. The next thing to do is to add all the classrooms' and offices' locations to the database.
% Checkpoint 1's testing phase will include testing the AR software on the footage.

\section*{Checkpoint 2}
The primary objective for Checkpoint 2 is to program the navigation software to show directions towards any valid location that a user will input.

\section*{Final Checkpoint}
The primary objective for the Final Checkpoint is to clean up the user interface for readability and usability and optimize routes for accessibility and efficiency.

\end{document}